% Created 2022-02-18 Fri 09:26
% Intended LaTeX compiler: pdflatex
\documentclass[11pt]{article}
\usepackage[utf8]{inputenc}
\usepackage[T1]{fontenc}
\usepackage{graphicx}
\usepackage{longtable}
\usepackage{wrapfig}
\usepackage{rotating}
\usepackage[normalem]{ulem}
\usepackage{amsmath}
\usepackage{amssymb}
\usepackage{capt-of}
\usepackage{hyperref}
\date{\today}
\title{}
\hypersetup{
 pdfauthor={},
 pdftitle={},
 pdfkeywords={},
 pdfsubject={},
 pdfcreator={Emacs 27.2 (Org mode 9.4.6)}, 
 pdflang={English}}
\begin{document}

\tableofcontents

\section{Using structs to structure related data}
\label{sec:org10c879e}
\begin{itemize}
\item so, it says that some of the best things about structs are the methods and type checking that go along with them
\item 
\end{itemize}
\section{5.1 defining and instantiating structs}
\label{sec:orgbc42997}
\begin{verbatim}
fn main() {

\end{verbatim}

\subsection{Structs: More flexible than tuples}
\label{sec:org3b660d8}
\begin{itemize}
\item every value is named
\item these named values make up the struct's \emph{fields}
\end{itemize}

\begin{verbatim}
struct User {
    active: bool,
    username: String,
    email: String,
    sign_in_count: u64,
}



let user1 = User {
    email: String::from("abc@123.com.ru.rs.sc.am.y.ou"), 
    username: String::from("someone"),
    active: true,
    sign_in_count: 1,
};
\end{verbatim}
\begin{itemize}
\item Note that the lines between struct fields are separated by commas, because it isn't a statement(";") or expression("<nothing>")
\end{itemize}



\subsection{mutability}
\label{sec:orgd24cf06}
\begin{itemize}
\item either all of the instance of a struct is mutable, or none of it is.
\item you can't declare individual fields to be mutable
\end{itemize}
\begin{verbatim}

let mut user1 = User {
    email: String::from("abc@123.com.ru.rs.sc.am.y.ou"), 
    username: String::from("someone"),
    active: true,
    sign_in_count: 1,
};
// now it can be mutated
user1.email = String::from("abababa@example.com");
\end{verbatim}
\#+end\_src
\subsection{using a function}
\label{sec:orgae9ce4f}
\begin{itemize}
\item In the build\_user function below, having the User\{\} part as the last \textbf{expression} (aka no semicolon) of the function means (implicitly) that it is the value to be returned
\begin{verbatim}
fn build_user(email: String, username: String) -> User {
    User {
        email: email,
        username: username,
        active: true,
        sign_in_count: 1,
    }
}
\end{verbatim}
\end{itemize}
\subsection{Ok, now the field init shorthand!!!!!!!!!!}
\label{sec:org1acc38a}
\begin{itemize}
\item As long as the named parameters in a function that instantiates a struct are the same as the struct's fields, you can use the \texttt{field init shorthand} syntax
\item saves time, less repetitive
\item Also, syntactically, you can give the last line a comma
\end{itemize}
\begin{verbatim}
fn build_user(email: String, username: String) -> User {
    User {
        email,
        username,
        active: true,
        sign_in_count: 1,
    }
}
\end{verbatim}
\begin{itemize}
\item so above, some of the fields are done by field init and the others are instantiated within the function
\end{itemize}
\subsection{\texttt{Struct Update} syntax: creating instances from other instances}
\label{sec:org38b217f}
\subsubsection{With and Without \texttt{Struct update} syntax}
\label{sec:org69e228b}
\begin{verbatim}
// without Struct Update syntax - manually declaring values
// let user2 = User {
//     active: user1.active,
//     username: user1.username,
//     email: String::from("another@example.com"),
//     sign_in_count: user1.sign_in_count,
// };

// With Struct update syntax - showing how fields not explicitly set should take their values from the given other instance
    // BUT! below no es bueno because the ownership of user1's "username" value was not transferred
let user2 = User {
    email: String::from("jack@jack.com"),
    ..user1
};

\end{verbatim}
\begin{enumerate}
\item SERIOUSLY important information re: above
\label{sec:orgb1b206d}
\begin{itemize}
\item OK: I think it is because the email and username fields are of \texttt{String} type, they therefore aren't string literals
\item aka since they're not string literals, they are just pointers to the same
\item I tried a few things to make it work but no bueno so far. Not sure why you'd ever want the use case where you could still have it in scope in a different User object, though
\item importantly, the user2 example above \textbf{moves} the ownership of user1's \texttt{username} field into user2
\item it is therefore out of scope for user
\end{itemize}
\end{enumerate}
\subsection{Using Tuple Structs w/o named fields to create different types}
\label{sec:orgf5db4d2}
\begin{itemize}
\item Ok, so the below is useful/important because now there are different types for two different pieces of data that would otherwise share the exact same structure
\item this way, functions that are supposed to take structs of type \texttt{Color} won't work if given a type \texttt{Point}
\end{itemize}
\begin{verbatim}
struct Color(i32,i32,i32);
struct Point(i32,i32,i32);

let black = Color(0,0,0);
let origin = Point(0,0,0);
\end{verbatim}
\subsection{unit-like structs without any fields}
\label{sec:org16da983}
\begin{itemize}
\item These behave similarly to \texttt{()} (c.f. ch3.2)
\item "Useful in situations in which you need to implement a trait on some type but don;t have any data taht you want to store inthe type itself"
\end{itemize}
\begin{verbatim}
struct AlwaysEqual;
let subject = AlwaysEqual;
\end{verbatim}
\begin{itemize}
\item implications of this will be covered more in ch. 10
\end{itemize}
\subsection{Ownership of struct data}
\label{sec:orgdea9345}
\begin{itemize}
\item this box is talking more about stuff we'll go into more depth in later
\item This is important, but I don't wnat to type it out right now so here's the link
\item \href{https://doc.rust-lang.org/book/ch05-01-defining-structs.html\#creating-instances-from-other-instances-with-struct-update-syntax}{go to the bottom of the page}
\end{itemize}
\subsection{close main}
\label{sec:orgb34faba}
\begin{verbatim}
}
\end{verbatim}
\section{5.2 example program using structs}
\label{sec:org23be68c}
\begin{verbatim}
fn main() {

\end{verbatim}
\subsection{beginning of the thing, without structs}
\label{sec:org3fb9a63}
"we'll start with a single variable, and then refactor until we're using structs instead"
\begin{verbatim}
let width1 = 30;
let height1 = 50;

println!(
    "the area of the rectangle is {} square pixels",
    area(width1, height1)
);

fn area(width: u32, height: u32) -> u32 {
    width * height
}
\end{verbatim}
\begin{itemize}
\item There are some interesting design points in this section
\item they say that "the area function is supposed to calculate the area of one rectangle, but we've wrote a function with two parameters"
\item since the parameters are related, eg they always only describe one rectangle, it would be better practice for legibility to group them together
\end{itemize}
\subsection{grouping using tuples}
\label{sec:orgb49cd7c}
\begin{verbatim}
let rect1 = (30, 50);
println!(
    "the area of the rectangle is {} square pixels",
    area(width1, height1)
);

fn area(dimensions: (u32, u32)) -> u32 {
    dimensions.0 * dimensions.1
}
\end{verbatim}
\begin{itemize}
\item so, this is more structured but is also in a way less clear
\item tuples don't name their fields
\item so it's simple in this example, but would get very complicated in others
\end{itemize}
\subsection{refactoring with structs}
\label{sec:org0f37588}
\begin{verbatim}
struct Rectangle {
    width: u32,
    height: u32,
}

let rect1 = Rectangle {
    width: 30,
    height: 50,
};

println!(
    "The rectangle's area is: {} square pixels",
    area(&rect1)
);

fn area(rectangle: &Rectangle) -> u32 {
    rectangle.width * rectangle.height
}
\end{verbatim}
\begin{itemize}
\item ok cool, so here we make the rectangle struct and then create an instance of it.
\item when we call the area function on it, we \textbf{borrow} the instance rect1
\item the area function defined knows to expect a \textbf{borrowed} Rectangle struct
\begin{itemize}
\item this parameter in the area function is further defined to be an \textbf{immutable borrow}
\end{itemize}
\item this is definitely more clear, as we can then call rectangle.width and rectangle.height
\end{itemize}

\subsection{Adding useful functionality with derived traits}
\label{sec:org17d3b48}
\begin{itemize}
\item so, we can't currently print a rect1 struct
\item we need to define a Display formatting
\item we would ALSO need to define a Debug formatting for our new struct Rectangle
\item adding the debugging is easy -> you effectively have to opt in
\end{itemize}
\begin{verbatim}
#[derive(Debug)]
struct Rectangle {
    width: u32,
    height: u32,
    }
fn main() {
    let rect1 = Rectangle {
        width: 30,
        height: 50,
        };
    // prints the whole structure
    println!("rect1 is {:?}",rect1); 
    // prints it in a line-by-line way
    println!("rect1 is {:#?}",rect1);
    }
\end{verbatim}
\begin{itemize}
\item debug (called by \texttt{\{:?\}}) effectively allows the entire instance to be printed as it is defined in the code
\item shows all fields of the instance
\item with larger structs we can use \texttt{\{:\#?\}} and fields will all be printed on separate lines
\end{itemize}

\subsection{dbg! macro}
\label{sec:org1482ff7}
\begin{itemize}
\item \texttt{dbg!} macro:
\begin{itemize}
\item takes ownership of an expression
\item prints the file and line number where the \texttt{dbg!} macro was
\item prints the value of the expression
\item returns ownership of the value
\item then prints this value to the \texttt{stderr} stream
\end{itemize}
\item this printing to \texttt{stderr} is different from how println! prints to \texttt{stdout}
\end{itemize}
\begin{verbatim}
#[derive(Debug)]
struct Rectangle {
    width: u32,
    height: u32,
}

fn main() {
    let scale = 2;
    let rect1 = Rectangle {
        width: dbg!(30*scale),
        height: 50,
    };
    dbg!(&rect1);
}
\end{verbatim}
\begin{itemize}
\item in the example above, we can use \texttt{dbg!} to
\end{itemize}
\subsection{close main}
\label{sec:orgad1fe96}
\begin{verbatim}
}
\end{verbatim}
\end{document}